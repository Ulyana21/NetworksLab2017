\documentclass[a4paper,14pt]{article}
\usepackage{amsmath}
\usepackage[utf8]{inputenc} % 
\usepackage{graphicx}
\graphicspath{{pic/}}
\DeclareGraphicsExtensions{.png,.jpg}
\usepackage{multicol}

\usepackage[russian]{babel} % правила переноса
\usepackage[left=2cm,right=2cm,
top=2cm,bottom=2cm,bindingoffset=0cm]{geometry} % для изменения размеров полей документа
\usepackage{commath}
\usepackage{listings}
\usepackage[framed,numbered,autolinebreaks,useliterate]{mcode}
\usepackage{longtable}
\begin{document}

%%%%%%%%%%%%%%%%%%%%%% Титульный лист %%%%%%%%%%%%%%%%%%%%%%

\begin{titlepage}
	\newpage
	\begin{center} % Размещение ткста - по центру
		Санкт-Петербургский политехнический\\ 
		университет Петра Великого\\
		Институт компьютерных наук и технологий\\
		Кафедра компьютерных систем и программных технологий\\
		\vspace{7cm}
		\textbf {Отчёт по лабораторной работе}\\
		\textbf {Дисциплина:} Сети и телекоммуникационные технологии\\
		\textbf{Тема:} Организация сетевого взаимодействия. Протокол TCP
	\end{center} % Конец размещения
	\vspace{8cm} % 
	
	\vfill
	
	\flushleft{Выполнил студент группы 43501/1} 
	\hfill\parbox{9 cm}{\hspace*{3cm}\hbox to 0cm{\raisebox{-1em}{\small(подпись)}}\hspace*{-0.8cm}\rule{3cm}{0.8pt} У.А. Пеньевская}\\[0.6cm]
	
	\flushleft{Преподаватель} \hfill\parbox{9 cm}{\hspace*{3cm}\hbox to 0cm{\raisebox{-1em}{\small(подпись)}}\hspace*{-0.8cm}\rule{3cm}{0.8pt} А.О.Алексюк }\\[0.6cm]
	
	\hfill\parbox{9 cm}{\hspace*{5cm} \today }\\[0.6cm]

	\vspace{\fill}
	\begin{center}
		Санкт-Петербург \\ 2017	
	\end{center}
\end{titlepage}

\tableofcontents

\newpage

\setcounter {section}{0}
\setcounter {equation}{0}
\setcounter {figure}{0}
\section{Цель работы}
\hspace{0,5cm}   Изучение принципов программирования сокетов с использованием протокола TCP.
\section{Индивидуальное задание}
Разработать распределенную систему, состоящую из приложений клиента и сервера, для распределенного выставления/просмотра курсов валют. Информационная система должна обеспечивать параллельную работу нескольких клиентов.

Основные возможности:
Серверное приложение должно реализовывать следующие функции:
\begin{enumerate}
\item Прослушивание определенного порта
\item Обработка запросов на подключение по этому порту клиентов
\item Поддержка одновременной работы нескольких клиентов с использованием механизма нитей и средств синхронизации доступа к разделяемым между нитями ресурсам.
\item Принудительное отключение конкретного клиента
\item Добавление новой валюты (кода валюты)
\item Удаление валюты
\item Добавление курса конкретной валюты
\item Выдача пользователю списка имеющихся валют с текущими курсами и абсолютными/относительными приращениями к предыдущим значениям
\item Выдача пользователю истории курса конкретной валюты
\item *Сохранение состояния при выключении сервера
\end{enumerate}
Клиентское приложение должно реализовывать следующие функции:
	\begin{enumerate}
\item Возможность параллельной работы нескольких клиентов с одного или нескольких IP-адресов
\item Установление соединения с сервером (возможно, с регистрацией на сервере)
\item Разрыв соединения
\item Обработка ситуации отключения сервером
\item Получение и вывод списка валют с котировками/изменениями
\item Передача команды на добавление валюты
\item Передача команды на удаление валюты
\item Передача команды на добавление курса валюты
\item Получение и вывод истории котировок валюты
\end{enumerate}
Настройка приложений:

Разработанное клиентское приложение должно предоставлять пользователю возможность задания IP-адреса или доменного имени сервера, а также номера порта сервера.

Тестирование:
Для тестирования запускается сервер системы котировок валют и несколько клиентов. В процессе тестирования проверяются основные функциональные возможности разработанной системы.
\section{Разработанный прикладной протокол}
Протокол TCP имеет следующий шаблон сообщения:
\begin{center}
	\textbf{<команда> <аттрибут> <аттрибут>}
\end{center}
В начале сообщения, всегда присутствует тип команды, далее взависимости от команды могут идти(взависимости от типа команды) аттрибуты, которые отделены друг от друга пробелом.
\begin{longtable}[H]{ccccc}
	\hline
	\multicolumn{1}{|c|}{}  & \multicolumn{1}{c|}{Код команды} & \multicolumn{1}{c|}{Аттрибуты}  & \multicolumn{1}{c|}{Действия}  &    \multicolumn{1}{c|}{Ответ сервера}                        \\ \hline
	
	\multicolumn{1}{|c|}{1} & \multicolumn{1}{c|}{A} & \multicolumn{1}{l|}{\begin{tabular}[c]{@{}l@{}}<ID>\\ <NAME>\\ <SHORT NAME>\\ <COURCE>  \end{tabular}} & \multicolumn{1}{l|}{\begin{tabular}[c]{@{}l@{}}Добавление новой валюты \\Add New Finance Position  \end{tabular}} & \multicolumn{1}{l|}{\begin{tabular}[c]{@{}l@{}}Added Finance Data: \\ ID: Name: Short name: \\Course: absCource: conCource  \end{tabular}} \\ \hline
	
	\multicolumn{1}{|c|}{2} & \multicolumn{1}{c|}{D} & \multicolumn{1}{l|}{\begin{tabular}[c]{@{}l@{}}<ID>  \end{tabular}} & \multicolumn{1}{l|}{\begin{tabular}[c]{@{}l@{}} Удаление валюты \\Delete Finance Position  \end{tabular}} & \multicolumn{1}{l|}{\begin{tabular}[c]{@{}l@{}}Finance Data: \\ ID:0 Name: Short name: \\Course: 0.00 absCource: 0.00\\ conCource: 0.00  \end{tabular}} \\ \hline
	
	\multicolumn{1}{|c|}{3} & \multicolumn{1}{c|}{R} & \multicolumn{1}{l|}{\begin{tabular}[c]{@{}l@{}}Отсутствуют  \end{tabular}} & \multicolumn{1}{l|}{\begin{tabular}[c]{@{}l@{}} Выдача пользователю списка \\ имеющихся  валют \\ с текущими курсами \\ и абсолютными/относительными \\ приращениями \\ к предыдущим значениям \\Read All Finance Courcess  \end{tabular}} & \multicolumn{1}{l|}{\begin{tabular}[c]{@{}l@{}}Finance Data: \\ ID:0 Name: Short name: \\Course: 0.00 \\ absCource: 0.00 conCource: 0.00  \end{tabular}} \\ \hline
	
	\multicolumn{1}{|c|}{4} & \multicolumn{1}{c|}{С} & \multicolumn{1}{l|}{\begin{tabular}[c]{@{}l@{}}<ID> \\<COURCE>   \end{tabular}} & \multicolumn{1}{l|}{\begin{tabular}[c]{@{}l@{}} Добавление курса конкретной валюты \\Add Finance Cource  \end{tabular}} & \multicolumn{1}{l|}{\begin{tabular}[c]{@{}l@{}}Finance Data: \\ ID:0 Name: Short name: \\Course: 0.00 absCource: 0.00\\ conCource: 0.00  \end{tabular}} \\ \hline
	
	\multicolumn{1}{|c|}{5} & \multicolumn{1}{c|}{O} & \multicolumn{1}{l|}{\begin{tabular}[c]{@{}l@{}} Отсутствуют   \end{tabular}} & \multicolumn{1}{l|}{\begin{tabular}[c]{@{}l@{}} Разрыв соединения \\ Exit Programm \end{tabular}} & \multicolumn{1}{l|}{\begin{tabular}[c]{@{}l@{}} -disconnect   \end{tabular}} \\ \hline
	
	\multicolumn{1}{|c|}{6} & \multicolumn{1}{c|}{M} & \multicolumn{1}{l|}{\begin{tabular}[c]{@{}l@{}} Отсутствуют   \end{tabular}} & \multicolumn{1}{l|}{\begin{tabular}[c]{@{}l@{}} Вывод меню \\  Show Menu \end{tabular}} & \multicolumn{1}{l|}{\begin{tabular}[c]{@{}l@{}} -   \end{tabular}} \\ \hline
	
	\multicolumn{1}{|c|}{7} & \multicolumn{1}{c|}{H} & \multicolumn{1}{l|}{\begin{tabular}[c]{@{}l@{}} <ID>   \end{tabular}} & \multicolumn{1}{l|}{\begin{tabular}[c]{@{}l@{}} Получение и вывод истории \\ котировок валюты \\ Show History  \end{tabular}} & \multicolumn{1}{l|}{\begin{tabular}[c]{@{}l@{}} -  \end{tabular}} \\ \hline
	
	\caption{Команды клиента}
\end{longtable}
Список команд, которыми оперирует клиент:
\begin{lstlisting}
| Add New Finance Position:  | A <ID> <NAME> <SHORT NAME> <COURCE>    
| Delete Finance Position:   | D <ID>  
| Read All Finance Courcess: | R                        
| Add Finance Cource:        | C <ID> <COURCE>  
| Exit Programm:             | O 
| Show Menu:                 | M          
| Show History               | H <ID>  
\end{lstlisting}
Посылка от сервера – клиенту.
Посылка состоит из символа типа char – команды, а также, при необходимости, - данных.
\begin{lstlisting}
C - Client List 
k <Socket number> - Kill klient 
\end{lstlisting}
\subsection{Описание структуры приложения}
Сервер:\\
Функция main:
	\begin{itemize}		
\item Инициализация всех используемых переменных;
\item Запуск событий, если таковые уже имеются;
\item Создание сокета;
\item Создание потока для прослушивания сокета;
\item Цикл чтения команд.
	\end{itemize}
 Поток для прослушивания сокета:
 \begin{itemize}
\item Прослушивание сокета;
\item Добавление подключенного клиента в список;
\item Создание сокета для работы с клиентом;
\item Создание потока для работы с клиентом.
	\end{itemize}
Цикл чтения команд:
 \begin{itemize}
\item Чтение команды из стандартного ввода;
\item Реакция на введенную команду (при корректном вводе команды) или игнорирование команды.
\end{itemize}
Поток для работы с клиентом:
 \begin{itemize}
\item Получение сообщения от клиента;
\item Анализ сообщения – является ли сообщение командой;
\item Ответ на команду или вывод сообщения с пометкой отправителя.
\end{itemize}
Клиент:

Функция main:
	\begin{itemize}	
\item Чтение ip адреса сервера;
\item Подключение к серверу;
\item Создание потока для получения данных от сервера;
\item Цикл чтения данных и отправка серверу.
\end{itemize}
Поток для получения данных от сервера:
 \begin{itemize}
\item Побайтовое получение данных от сервера;
\item Реакция на полученные данные .
	\end{itemize}
\section{Реализация программы}
\subsection{Структура проекта}
При разработке приложения для операционной системы семейства Windows использовалась среда разработки Microsoft Visual Studio.

Язык программирования — С++.

\subsection{Сетевая часть TCP}
Клиентское приложение в TCP только отсылает команды на сервер, поэтому оно ничем не отличается от telnet клиента. Сервер обрабатывает команды, работает с коллекциями, сохраняет и загружает свое состояние, присылает уведомления и др. Делаем вывод, что клиентская программа потребляет ничтожно малый процессорный ресурс, в то время как сервер - наоборот.

На сервере, в первую очередь, происходит инициализация WinSock (на Windows), создание сокета (функция socket), привязка сокета к конкретному адресу (функция bind). Реализация инициализации сервера представлена в \textbf{Приложении 1.}

После этого ожидаем подключения клиентов в бесконечном цикле, с помощью функции accept. Если функция возвращает положительное значение, которое является клиентским сокетом, то создаем новый поток, в котором обрабатываем клиентские сообщения. Реализация подключения клиентов представлена в \textbf{Приложении 2.}

Клиентский поток вызывает функцию считывания символов в бесконечном цикле, как только при считывается знак перевода строки, функция возращает прочитанные символы. Если функция не вернула исключение, то посылаем команду на обработку, в противном случае это обозначает отключение клиента. Также отключение клиента может быть произведено извне обработчика клиентского потока, посредством закрытия клиентского сокета(функция считывания в этом случае сразу же вернет исключение) командой \textbf{k <Socket number>} (Kill klient). Реализация клиентского потока
представлена в \textbf{Приложении 3}.

\section{Тестирование}
\subsection{Тестирование серверного приложения}
Запускаем клиент-серверное приложение
\begin{figure}[h!]
	\noindent\centering{
		\includegraphics[width=170mm]{1}}
	\caption{Клиент-серверное приложение "Курс валют"}
	\label{figCurves}
\end{figure}

Запускаем еще одного клиента

\begin{figure}[h!]
	\noindent\centering{
		\includegraphics[width=170mm]{2}}
	\caption{Подключение второго клиента}
	\label{figCurves}
\end{figure}

Список подключенных клиентов можно посмотреть с помощью команды - С

\begin{figure}[h!]
	\noindent\centering{
		\includegraphics[width=70mm]{3}}
	\caption{Список подключенных клиентов}
	\label{figCurves}
\end{figure}

Отключить клиента - k <Socket number>

\begin{figure}[h!]
	\noindent\centering{
		\includegraphics[width=70mm]{4}}
	\caption{Отключение одного из клиентов}
	\label{figCurves}
\end{figure}



\newpage

\subsection{Тестирование клиентского приложения}

Проверка работоспособности команд

Разорвем соединение с помощью команды O (Exit programm)
\begin{figure}[h!]
	\noindent\centering{
		\includegraphics[width=120mm]{21}}
	\caption{Разрыв соединения}
	\label{figCurves}
\end{figure}

Добавим новую валюту вводом команды: A 1 RUB RU 70.23

\begin{figure}[h!]
	\noindent\centering{
		\includegraphics[width=150mm]{5}}
	\caption{Добавление новой валюты}
	\label{figCurves}
\end{figure}

Сервер выполнил команду добавление валюты (Added Finance Data)\\

Изменим курс валюты с помощью команды: A 1 30
\begin{figure}[h!]
	\noindent\centering{
		\includegraphics[width=140mm]{6}}
	\caption{Изменение курса валюты}
	\label{figCurves}
\end{figure}

На сервере получили сообщение о том, что курс валюты изменился. 

absCourse = 70.23-30=40.23

Добавим еще одну валюту (A 2 USA US 1.23), удалим её (D 2) и прочитаем список всех валют (R)
\begin{figure}[h!]
	\noindent\centering{
		\includegraphics[width=140mm]{7}}
	\caption{Удаление валюты}
	\label{figCurves}
\end{figure}
\newpage

Вывести меню еще раз командой М

\begin{figure}[h!]
	\noindent\centering{
		\includegraphics[width=160mm]{8}}
	\caption{Вывод меню}
	\label{figCurves}
\end{figure}

Вывести историю изменения валюты командой H <ID>

\begin{figure}[h!]
	\noindent\centering{
		\includegraphics[width=40mm]{9}}
	\caption{Вывод истории}
	\label{figCurves}
\end{figure}

\subsection {Обработка ошибок и предупреждений клиентского приложения}

\subsubsection {Неполный список аргументов}
Исходные данные: A 1

\begin{figure}[h!]
	\noindent\centering{
		\includegraphics[width=170mm]{15}}
	\caption{Обработка неполного количества аргументов}
	\label{figCurves}
\end{figure}

\subsubsection {Избыток цифр в ID}
Исходные данные: A 231241214 RUB RU 12
\begin{figure}[h!]
	\noindent\centering{
		\includegraphics[width=170mm]{16}}
	\caption{Обработка избытка цифр в ID}
	\label{figCurves}
\end{figure}

\subsubsection {Некорректный ID}
Исходные данные: A 34 RUB RU 12
\begin{figure}[h!]
	\noindent\centering{
		\includegraphics[width=170mm]{17}}
	\caption{Обработка некорректного ID}
	\label{figCurves}
\end{figure}

\subsubsection {Слишком длинное название <NAME>}
Исходные данные: A 2 ruuuuuuuuuuuuuuuuuuuuuuuuuuuuuuuuuuuuuuuuuuub ru 12

\begin{figure}[h!]
	\noindent\centering{
		\includegraphics[width=170mm]{10}}
	\caption{Обработка <Name>}
	\label{figCurves}
\end{figure}

\subsubsection {Слишком длинное <SHORT NAME>}
Исходные данные: A 2 RUB ruuuuuuuuuuuuuuuuuuu 12

\begin{figure}[h!]
	\noindent\centering{
		\includegraphics[width=170mm]{10}}
	\caption{Обработка <Short name>}
	\label{figCurves}
\end{figure}

\subsubsection {Курс с таким ID не найден}
Исходные данные: D 4

\begin{figure}[h!]
	\noindent\centering{
		\includegraphics[width=170mm]{14}}
	\caption{Обработка <Short name>}
	\label{figCurves}
\end{figure}
\section{Вывод}
В данной лабораторной работе были реализованы клиент-серверные программы выставления курса валют, с написанием собственного протокола на основе TCP. Протокол был реализован на языке C++ для операционной системы Windows.\\\\
В ходе выполнения работы были изучены принципы организации многопоточного сервера, изучены принципы синхронизации доступа к глобальным переменным.
В разработанном в ходе работы сервере для каждого клиента создается отдельный поток. Такой подход оправдан, т.к. клиенты могут исполнять долгие операции и операции различной трудоемкости. В этом случае использование отдельного потока для каждого клиента обеспечивает минимизацию взаимного влияния клиентов друг на друга. Создание потока – достаточно ресурсоемкая операция как по времени выполнения, так и по расходу оперативной памяти, поэтому при использовании данного подхода нужно следить за расходом памяти. \\\\


\textbf{Приложение 1}
\begin{lstlisting}
DWORD thID;

WSADATA wsaData;
int iResult; 
int length=0;

struct addrinfo *result = NULL;
struct addrinfo hints;

printf("Mail Server Started...\n");

iResult = WSAStartup(MAKEWORD(2,2), &wsaData);
printf("WSAS starting... \n");
if (iResult != 0) 
{
	printf("WSAStartup failed with error: %d\n", iResult);
	return 1;
}

ZeroMemory(&hints, sizeof(hints));

sockaddr_in Server; 
Server.sin_family = AF_INET;
Server.sin_addr.s_addr = inet_addr("127.0.0.1"); //local_addr.sin_addr.s_addr=0; 
Server.sin_port = htons(DEFAULT_PORT); 

SOCKET MySocket;

MySocket = socket(AF_UNSPEC, SOCK_STREAM, IPPROTO_TCP);
printf("Listening starting... \n");
if (MySocket == INVALID_SOCKET) 
{
	printf("socket failed with error: %ld\n", WSAGetLastError());
	freeaddrinfo(result);
	WSACleanup();
	return 1;
}
printf("Success! \n");

iResult = bind( MySocket, (SOCKADDR *) & Server, sizeof (Server));
printf("Bind starting... \n");
if (iResult == SOCKET_ERROR) {
	printf("bind failed with error: %d\n", WSAGetLastError());
	freeaddrinfo(result);
	closesocket(MySocket);
	WSACleanup();
	return 1;
}
\end{lstlisting}

\textbf{Приложение 2}
\begin{lstlisting}
DWORD WINAPI ServToClient(LPVOID client_socket)
{
	SOCKET my_sock;
	my_sock=((SOCKET *) client_socket)[0];
	int MyID = ID;
	char ID[8];
	char SockBuf[8];
	_itoa_s(my_sock, SockBuf, 10);
	_itoa_s(MyID, ID, 10);
	strcpy_s(UserInfo[MyID], "ID: ");
	strcat_s(UserInfo[MyID], ID);
	strcat_s(UserInfo[MyID], " Addr: ");
	strcat_s(UserInfo[MyID], inet_ntoa(client_addr.sin_addr));
	strcat_s(UserInfo[MyID], " S: ");
	strcat_s(UserInfo[MyID], SockBuf);
	strcat_s(UserInfo[MyID], "\n");
	printf("%s", UserInfo[MyID]);
	fflush(stdout);

	char recvbuf[DEFAULT_BUFLEN]; 
	int iResult;
	char user[30];

	memset(recvbuf, 0, DEFAULT_BUFLEN);
	int end = 0;
	memset(user, 0, 30);
	
	bool r = false;
		
	CCount = CCount -1; // Umenshaem schetchik klientov
	printf("-disconnect\n"); PRINTNUSERS
		
	// cleanup
		
	// WSACleanup();
		
	printf("Complited! \n");
		
	return 0;
}
\end{lstlisting}
\textbf{Приложение 3}
\begin{lstlisting}
  iResult = WSAStartup(MAKEWORD(2,2), &wsaData);
printf("WSAS starting... \n");
if (iResult != 0) {
	printf("WSAStartup failed with error: %d\n", iResult);
	return 1;
}

SOCKET my_sock;
my_sock = socket(AF_UNSPEC, SOCK_STREAM, IPPROTO_TCP);
printf("Opening socket... \n");
if (my_sock == INVALID_SOCKET) 
{
	printf("socket failed with error: %ld\n", WSAGetLastError());
	WSACleanup();
	return 1;
}
printf("Success! \n");

if (inet_addr(SERVERADDR)!=INADDR_NONE)
client.sin_addr.s_addr=inet_addr(SERVERADDR);
else

if (hst=gethostbyname(SERVERADDR))
{
	((unsigned long *)&client.sin_addr)[0]=((unsigned long **)hst->h_addr_list)[0][0];
}
else 
{
	printf("Invalid address %s\n",SERVERADDR);
	closesocket(my_sock);
	WSACleanup();
	return -1;
}

iResult = connect( my_sock, (SOCKADDR *) & client, sizeof(client));
printf("Connecting... \n");
if (iResult == SOCKET_ERROR) 
{
	closesocket(my_sock);
	my_sock = INVALID_SOCKET;
}
if (my_sock == INVALID_SOCKET) 
{
	printf("Unable to connect to server!\n");
	WSACleanup();
	return 1;
}
printf("Connected to %s \n",SERVERADDR);
\end{lstlisting}
\end{document}
